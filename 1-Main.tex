
\documentclass[sigchi]{acmart}

\usepackage[colorinlisttodo]{todonotes}

\usepackage{listings}
\usepackage{xcolor}

\definecolor{codegreen}{rgb}{0,0.6,0}
\definecolor{codegray}{rgb}{0.5,0.5,0.5}
\definecolor{codepurple}{rgb}{0.58,0,0.82}
\definecolor{backcolour}{rgb}{0.95,0.95,0.92}

\lstdefinestyle{mystyle}{
    backgroundcolor=\color{backcolour},   
    commentstyle=\color{codegreen},
    keywordstyle=\color{magenta},
    numberstyle=\tiny\color{codegray},
    stringstyle=\color{codepurple},
    basicstyle=\ttfamily\footnotesize,
    breakatwhitespace=false,         
    breaklines=true,                 
    captionpos=b,                    
    keepspaces=true,                 
    numbers=left,                    
    numbersep=5pt,                  
    showspaces=false,                
    showstringspaces=false,
    showtabs=false,                  
    tabsize=2
}

\lstset{style=mystyle}

\DeclareTextFontCommand{\mytexttt}{\ttfamily\hyphenchar\font=45\relax}

\copyrightyear{2020}
\acmYear{2020}
\setcopyright{acmlicensed}
\acmConference[Workshop XXX '2020]{A Workshop on Software Engineering}{2020}{Somewhere}
\acmBooktitle{Workshop XXX '2020: Workshop on Software Engineering, Somewhere}
%\acmPrice{15.00}
%\acmDOI{10.1145/1122445.1122456}
%\acmISBN{978-1-4503-9999-9/18/06}

\begin{document}

%%
%% The "title" command has an optional parameter,
%% allowing the author to define a "short title" to be used in page headers.
\title{PAMELA: an annotation-based Java Modelling Framework}

%%
%% The "author" command and its associated commands are used to define
%% the authors and their affiliations.
%% Of note is the shared affiliation of the first two authors, and the
%% "authornote" and "authornotemark" commands
%% used to denote shared contribution to the research.
\author{Sylvain Guérin, Guillaume Polet, Caine Silva, Joel Champeau}
%\orcid{1234-5678-9012}
\affiliation{%
  \institution{Lab-STICC \\ ENSTA Bretagne}
%  \streetaddress{P.O. Box 1212}
  \city{Brest}
  \country{France}
  %  \postcode{43017-6221}
}
\email{firstname.lastname@ensta-bretagne.fr}



%%
%% By default, the full list of authors will be used in the page
%% headers. Often, this list is too long, and will overlap
%% other information printed in the page headers. This command allows
%% the author to define a more concise list
%% of authors' names for this purpose.
\renewcommand{\shortauthors}{S. Guérin, et al.}

%%
%% The abstract is a short summary of the work to be presented in the
%% article.

\begin{abstract}
This article presents PAMELA, an annotation-based Java Modelling framework. PAMELA provides a smooth integration between model and code, and enable Java developers to handle software development both at conceptual level and at source-code level, without code transformation and/or generation, and avoiding round-tripping issues.

% TODO...

% Parler aussi de Aspect Oriented Programming, Traits programming, Run-time weaving, Semantics Add-On, MOP (Meta Object Protocol) 

\end{abstract}



%%
%% The code below is generated by the tool at http://dl.acm.org/ccs.cfm.
%% Please copy and paste the code instead of the example below.
%%

%%
%% Keywords. The author(s) should pick words that accurately describe
%% the work being presented. Separate the keywords with commas.
\keywords{Model-Driven Engineering, Java Modelling Framework, Meta-modeling, Models@Runtime}


%%
%% This command processes the author and affiliation and title
%% information and builds the first part of the formatted document.
\maketitle

\input{3-Introduction}

\section{Approach overview}

We present in this section an overview for PAMELA approach.

\subsection{Model serialization in source code}

We advocate for a strong coupling between model and source-code, to give architects and developers a way to both interact during the whole development cycle. PAMELA is an annotation-based Java modeling framework providing a smooth integration between model and code, without code generation nor externalized model serialization. The idea is to avoid separation between modeling and code to facilitate consistency management and avoid round-tripping issues.

To do so, we argue that source code is the right artefact to encode the model with metadata information stored in tagged code. This requires an annotation-enabled language. Such language supports the attribute-oriented programming if its grammar allows adding custom declarative tags to annotate standard program elements. Java programming language from version 1.5 is a good candidate with the support of annotations.

\begin{figure}
    \centering
    \includegraphics[width=1.0 \columnwidth]{PamelaVision.pdf}
    \caption{PAMELA approach for modelling}
    \label{fig:PamelaVision}
\end{figure}

Figure \ref{fig:PamelaVision} shows PAMELA approach for storing model in source code. The model is inlined in many source code files, with a set of annotations covering PAMELA metamodel as presented in the next subsection.

\subsection{PAMELA use process}

Coupling model and code into the same artefact open new ways of programming. The classical way relies on \emph{programmers} that produce code reusing pre-existing modeling concepts. These concepts are implemented by \emph{modelers} that provides the right annotations the programmers use. This is, for instance, the process followed by JEE developers reusing JEE specific annotations. The evolution rhythm between models and code is low. This programming way is still possible with PAMELA, but we allow the ability to reach a high evolution rhythm when the programmer becomes also the modeler. In fact, when a pattern, an abstraction, a generalization is identified by the programmer, s/he can use PAMELA to develop and capitalize on this abstraction by increasing PAMELA metamodel. Our experience shows that, in that case, introducing and reusing new concepts (1) reduce the size of the code, (2) reduce the risk of errors and (3) improve the code structure. The cycle of development between the model and the code can then be drastically reduced, leading to what we call \emph{continuous modeling}.

The code size is reduced because abstractions factorize the recognize pattern and the previous code is replaced by the use of the abstraction at the right place. This also reduce the risk since the previous code is now generated by the PAMELA framework with all the required verification. And finally, the code structure is improved since it matches the way the programmer conceptualizes (models) her/his code. 

The developed metamodels are implemented by annotations that relies on Java/JVM entities and mechanisms. They include consistency checking that constrains their use and help the programmer. We have experimented their use with setter/getter to define POJO entities, with traits to implement multiple inheritance or roles and rules to set security rules on classes.

\subsection{PAMELA metamodel}

PAMELA metamodel is presented in figure \ref{fig:PamelaMetaModel}.
This metamodel is classical and reflects a common class diagram vision such as in UML. 

\begin{figure}
    \centering
    \includegraphics[width=1.0 \columnwidth]{PamelaMetaModel.pdf}
    \caption{PAMELA metamodel}
    \label{fig:PamelaMetaModel}
\end{figure}

\subsubsection{PAMELAModel}

A \texttt{PAMELAModel} is defined as a set of references to \texttt{ModelEntity}. 

\subsubsection{ModelEntity}

A \texttt{ModelEntity} reflects a concept and is encoded in a java \texttt{interface}. PAMELA metamodel allows multiple inheritance: thus \texttt{ModelEntity} may define a set of parent entities. A \texttt{ModelEntity} also defines some properties, encoded as \texttt{ModelProperty}. Note that reification of \texttt{ModelEntity} is performed in a java \texttt{interface} (and not a class), which only defines API whithout any implementation for methods. A partially implemented \texttt{abstract} java texttt{class} may be defined as partial base implementation (conform to implemented interface).

\subsubsection{ModelProperty}

A \texttt{ModelProperty} is identified by a name, a cardinality (simple or multiple) and a type, which can be a reference to another \texttt{ModelEntity}, or a Java type (a primitive or an arbitrary complex Java type). Depending on its cardinality, a \texttt{ModelProperty} is bound to a set of methods reflecting use of property.
\begin{itemize}
    \item A \emph{read-only single property} will define read-access of its value using a \emph{getter} (a java method defined in java interface taking no argument and returning desired value).
    \item A \emph{read-write single property} will define a \emph{getter} and a \emph{setter} (a java method taking value to be set as unique argument)
    \item A \emph{read-write multiple property} will define a \emph{getter}, a \emph{adder} (a java method taking value to be added as unique argument), a \emph{remover} (a java method taking value to be removed as unique argument), and may define additional methods for extended features such as reindexing for example.
\end{itemize}

A strong interest of the approach is that the model is encoded in java, and must be compiled. It forces the java compiler to perform required checks for a PAMELA model encoded in a strong typed program. Execution semantics of model is fully compatible with Java semantics. Many validation rules are automatically performed through classical java compilation, independently of underlying PAMELA execution semantics.

\subsection{A basic example}

The following code listing represents a very basic model with two entities \emph{Book} and \emph{Library}. Entity \emph{Book} defines two read-write single properties \emph{title} and \emph{ISBN} with single cardinality and with \texttt{String} type. Entity \emph{Book} also define a constructor with initial \emph{title} value. Entity \emph{Library} defines a read-write multiple properties \emph{books} referencing \emph{Book} instances. Note that this code is sufficient to execute the model, while no line of code is required (only java interface and API methods are declared here). 

%\begin{figure}
%    \centering
\begin{lstlisting}[language=Java,basicstyle=\ttfamily\footnotesize]
@ModelEntity
public interface Book 
  extends AccessibleProxyObject {
  @Initializer
  public Book init(@Parameter("title") 
    String aTitle);
  @Getter("title")
  public String getTitle();
  @Setter("title")
  public void setTitle(String aTitle);
  @Getter("ISBN")
  public String getISBN();
  @Setter("ISBN")
  public void setISBN(String value);
}

@ModelEntity
public interface Library
  extends AccessibleProxyObject {
  @Getter(value = "books", 
    cardinality = Cardinality.LIST)
  public List<Book> getBooks();
  @Adder("books")
  public void addToBooks(Book aBook);
  @Remover("books")
  public void removeFromBooks(Book aBook);
  @Reindexer("books")
  public void moveBookToIndex(Book aBook, 
    int index);
  @Finder(collection = "books",
    attribute = "title")
  public Book getBook(String title);
}
\end{lstlisting}
 %   \caption{A basic PAMELA model with two entities \texttt{Library} and \texttt{Book}}
 %   \label{fig:ABasicPamelaModel}
%\end{figure}

Execution of this model may be performed using following simple lines of code.

%\begin{figure}
%    \centering
\begin{lstlisting}[language=Java,basicstyle=\ttfamily\footnotesize]
// Instantiate the meta-model
// by computing the closure of concepts graph
ModelContext modelContext = ModelContextLibrary
  .getModelContext(Library.class);
// Instantiate the factory
ModelFactory factory 
  = new ModelFactory(modelContext);
// Instantiate a Library
Library myLibrary 
  = factory.newInstance(Library.class);
myLibrary.setName("My library");
// Instantiate some Books
Book myFirstBook = factory.newInstance(
  Book.class, "Lord of the ring");
Book anOtherBook = factory.newInstance(
  Book.class, "Holy bible");
myLibrary.addToBooks(myFirstBook);
myLibrary.addToBooks(anOtherBook);
\end{lstlisting}
 %   \caption{Executing a PAMELA model}
 %   \label{fig:ExecutingPamelaModel}
%\end{figure}

The first line of code instantiates a \texttt{ModelContext} by introspecting and computing the closure of concepts graph obtained while starting from \texttt{Library} entity and following \texttt{parentEntities} and \texttt{properties} relationships. This call builds at runtime a \emph{PAMELAModel}, while dynamically following links reflected by compiled byte-code. A factory \texttt{ModelFactory} is then instantiated using that \texttt{ModelContext}, allowing to instantiate \emph{Library} and \emph{Book} instances.

\subsection{Handling custom code}

A major challenge to be addressed by MDE (Model-Driven Engineering) approaches is the ability to integrate custom implementation to a base of code derived from a formal model. The major drawback is the way back (round-tripping).

PAMELA frameworks provides an elegant way to do it, while using common extension points such as inheritance, as offered by java language. Custom implementations should be declared in java classes, this or those classes declared as partial implementation(s) of related \emph{ModelEntity}. 

The following example shows how to integrate custom code to the fully interpreted \emph{Book} entity described above. The partial custom implementation is offered by a partial class (note the \texttt{abstract} keyword), declared in annotation header of model entity. Custom implementations are defined using classical java implementation/overrides scheme. Here we define implementation of \texttt{read()} method, which has no annotation (and thus unable to be processed by PAMELA framework), and also implementation of a custom getter for \emph{title}, returning a default value when no value is defined for that property. Note that this implementation references default interpretated implementation (call to \texttt{performSuperGetter(String)} method).

%\begin{figure}
%    \centering
\begin{lstlisting}[language=Java,basicstyle=\ttfamily\footnotesize]
@ModelEntity
@ImplementationClass(BookImpl.class)
public interface Book 
  extends AccessibleProxyObject {
  static final String TITLE = "title";
  @Getter(TITLE)
  String getTitle();
  // ... title property declarations ...
  void read();
}
// Provides a partial implementation for Book
public static abstract class BookImpl 
  implements Book {
  @Override
  public String getTitle() {
    String title = performSuperGetter(TITLE);
    if (title == null) {
      return "This book has no title";
    }
    return title;
  }
  @Override
    public void read() {
      // do the job
    }
}
\end{lstlisting}
 %   \caption{Executing a PAMELA model}
 %   \label{fig:ExecutingPamelaModel}
%\end{figure}

As said previously, PAMELA framework supports multiple inheritance. In this context, it is possible to provide multiple partial classes as implementation for a given \emph{ModelEntity}. To do so, we use abstract inner classes tagged with \texttt{@Implementation}, and the composition is made at run-time.

\subsection{Run-time considerations}

Resulting model execution is a combination of:
\begin{itemize}
    \item plain java byte-code, as the result of the basic compilation of source code,
    \item and an embedded PAMELA interpreter, executing semantics reflected by \emph{ModelEntity} and \emph{ModelProperty} declarations.
\end{itemize}
 
This composition offers many benefits: 
\begin{itemize}
    \item Strong coupling between model and code
    \item Strong typing is kept, and required checks are performed by the java compiler
    \item PAMELA framework provides interpretation of model@runtime
    \item No need to generate POJO (plain old java objects), as their execution follow the standard semantics (less code, less bugs)
    \item Custom implementation are provided if needed, using classical java extension points
    \item It offers a way to intercept method calls and instrument the code
\end{itemize}

From a technical point of view, PAMELA implementation uses \emph{javassist} reflection library, providing \texttt{MethodHandler} mechanism, which is a way to override the java dynamic binding. Invoking a method on an object which is part of a PAMELA model, caused the real implementation to be called when existing (more precisely dispatch code execution between all provided implementations), or the required interpretation according to underlying model to be executed. This provides also an extension point allowing to instrument the code, which is used for other features such as undo/redo stack management, and assertion checking at run-time (support for Design by Contract, aka JML).
% TODO: trouver des références pour ca

PAMELA framework is a 100\% pure java (> 1.5), compilable by a classical java compiler and executable in a classical Java Virtual Machine.

% Rajouter un paragraphe sur le fait que PAMELA est une implementation du MOP (meta object protocol) ?

\subsection{Common PAMELA annotations}

Here is a non-exhaustive list of the most common java annotations reflecting PAMELA metamodel.

\begin{itemize}
    \item \texttt{@ModelEntity}: tag annotating \texttt{interface} as \emph{ModelEntity}. May declares an abstract entity.
    \item \texttt{@ImplementationClass}: tag annotating \emph{ModelEntity} \texttt{interface} and precising abstract java \texttt{class} to be used as base implementation.
    \item \texttt{@Implementation}: tag annotating a partial implementation (abstract inner \texttt{class} defined in implemented \texttt{interface}), and used in the context of multiple inheritance.
    \item \texttt{@Getter(String)}: tag annotating method as unique getter for implicit \emph{ModelProperty} whose identifier is the declared String value. May also declares cardinality, enventual inverse property, default value and some other features.
    \item \texttt{@Setter(String)}: tag annotating method as unique setter for implicit \emph{ModelProperty} whose identifier is the declared String value.
    \item \texttt{@Adder(String)}: tag annotating method as unique adder for implicit multiple cardinality \emph{ModelProperty} whose identifier is the declared String value.
    \item \texttt{@Remover(String)}: tag annotating method as unique remover for implicit multiple cardinality \emph{ModelProperty} whose identifier is the declared String value.
    \item \texttt{@Reindexer(String)}: tag annotating method as unique reindexer for implicit multiple cardinality \emph{ModelProperty} whose identifier is the declared String value.
    \item \texttt{@Initializer}: tag annotating a method used as a constructor for related \emph{ModelEntity}
   \item \texttt{@Deleter}: tag annotating a method used as explicit destructor for related \emph{ModelEntity}
    \item \texttt{@Finder(String,String)}: tag annotating method as a fetching request for a given \emph{ModelProperty} with a given attribute
    \item \texttt{@CloningStrategy}: allows to customize cloning strategy for a given \emph{ModelProperty}
    \item \texttt{@Embedded}: allows to declare a given \emph{ModelProperty} as embedded according to PAMELA semantics
    \item \texttt{@Imports} and \texttt{@Imports}: allows to declare some entities to be included in infered metamodel while \texttt{ModelContext} computation.
    \item \texttt{@XMLElement} and \texttt{@XMLAttribute}: used to specify XML serialization for PAMELA instances
    
\end{itemize}






\section{PAMELA features}

This section presents an overview of major features available in the context of PAMELA framework use.

\subsection{Meta-model at run-time}

Metamodel closure is computed on runtime, working on the classpath of launched java application, and starting from a simple java interface (or a collection of java interfaces) which is/are PAMELA-annotated. From a mathematical point of view, internal representation of the metamodel is a graph whose vertex are PAMELA \emph{ModelEntities} (annotated java interface), and edges are either inheritance links or reference links (a property whose type is another \emph{ModelEntity}. \mytexttt{@Imports} and \mytexttt{@Import} annotations allows to include some other \emph{ModelEntities} in the metamodel. On the contrary, an annotation attribute \mytexttt{@Getter(...ignoreType=true)} allows to ignore the link. In that context, metamodel computation is a graph closure computation, starting from a collection of vertices. 

Metamodel closure computation on-the-fly provides an interesting approach to deal with model fragmentation.

% TODO: references

\subsection{PAMELA objects life-cycle management}

A Metamodel computation is represented by a \mytexttt{ModelContext} and uses a \mytexttt{ModelFactory} built with that model context to handle instances of that metamodel. The \mytexttt{ModelFactory} is responsible of the life-cycle of instances of metamodel (construction and destruction of Java instances). \mytexttt{@Initializer} annotation allows to define parametered constructors for \emph{ModelEntity} instances.

\mytexttt{AccessibleProxyObject} is a Java interface providing utilities methods which are interpreted by internal PAMELA interpreter. This includes calls to internal code execution, such as \mytexttt{performSuperSetter(String,Object)} which represent a call to internal setter of property identified by supplied String value.

Figure \ref{fig:LifeCycle} illustrates life-cycle of objects beeing instantiated as PAMELA instances. The \mytexttt{ModelFactory} initiates creation and triggers right constructor during a phase where the object is in \mytexttt{isCreating} status.

\begin{figure}
    \centering
    \includegraphics[width=0.9 \columnwidth]{LifeCycle.pdf}
    \caption{Life-cycle of PAMELA objects}
    \label{fig:LifeCycle}
\end{figure}

Modifications of objects are internally tracked by PAMELA interpreter which manages \mytexttt{Modified} status, according to containment semantics as presented further (a contained object modification implies object flagged as modified, and implied container flaggued as modified too). Saving object graph brings back object status in \mytexttt{Alive/Saved} status.

Since calls to any features of model objects are dispatched by the internal interpreter, PAMELA runtime offers a multi-level undo/redo stack tooling. When enabled, this scheme allows to store and manage an edition model composed of atomic edits. Figure \ref{fig:AtomicEditMetaModel} presents atomic edits metamodel for a fine-grained model modification tracking system. That mechanism provides undo/redo features, virtually unlimited. For performance reasons, we can set a maximum depth for undo/redo operations.

\begin{figure}
    \centering
    \includegraphics[width=1.0 \columnwidth]{AtomicEditMetaModel.pdf}
    \caption{Atomic edits metamodel}
    \label{fig:AtomicEditMetaModel}
\end{figure}

\mytexttt{DeletableProxyObject} provides delete (and undelete) features to \emph{ModelEntity} instances. Deletion features are performed using a context which is a graph closure computation for all the objects which have to be deleted. PAMELA also offers undelete feature which allow to resurrect a deleted object. Deleted objects are still resurectable until they are still in the undo/redo stack. A deleted object who is leaving scope of maximum depth of undo/redo stack is destroyed. Object is fully dereferenced, ready for garbage collecting, and cannot be resurrected again.

\subsection{Meta-programming support}

Figure \ref{fig:PamelaMetaModel} represents PAMELA metamodel. Concept of \emph{ModelProperty} represent access to a value (simple or with multiple cardinality). This read-access is implicitely implemented using \mytexttt{get:} protocol. Following the same logic, PAMELA properties may expose following protocols:
\begin{itemize}
    \item \mytexttt{get:} read-access to a data. This protocol is implemented by all kind of \emph{ModelProperties}.
    \item \mytexttt{set:} write-access to a data. This protocol is implemented by \emph{ModelProperties} implementing \mytexttt{SettablePropertyImplementation} API.
    \item \mytexttt{add:} and \mytexttt{add:AtIndex:} add-access to a multiple cardinality data. These protocols are implemented by \emph{ModelProperties} implementing \mytexttt{MultiplePropertyImplementation} API.
    \item \mytexttt{remove:} remove-access to a multiple cardinality data. This protocols is implemented by \emph{ModelProperties} implementing \mytexttt{MultiplePropertyImplementation} API.
    \item \mytexttt{delete:} and \mytexttt{undelete:} delete/undelete protocols, implemented by all \emph{ModelProperties}
\end{itemize}

Unless specified in \mytexttt{@PropertyImplementation} PAMELA annotations, default implementation are provided the PAMELA interpreter. Default implementations are encoded in well identified classes which are extendable. 

PAMELA allows programmers to define their custom implementation for a given set of properties, while providing a class implementing \mytexttt{@PropertyImplementation} API and some protocol implementation (depending on the nature of the \emph{ModelProperty}. Following excerpt of code illustrate use of two custom property implementations.

\begin{lstlisting}[language=Java,basicstyle=\ttfamily\footnotesize]
@ModelEntity
public interface MyConcept {

	static final String VALUE = "value";
	static final String SUB_CONCEPTS = "someSubConcepts";

	@Getter(value = VALUE)
	@PropertyImplementation(MyPropertyImplementation.class)
	String getValue();

	@Setter(VALUE)
	public void setValue(String value);

	@Getter(value = SUB_CONCEPTS, cardinality = Cardinality.LIST)
	@PropertyImplementation(MyListCardinalityPropertyImplementation.class)
	@Embedded
	List<MySubConcept> getSubConcepts();

	@Adder(SUB_CONCEPTS)
	void addToSubConcepts(MySubConcept subConcept);

	@Remover(SUB_CONCEPTS)
	void removeFromSubConcepts(MySubConcept subConcept);
}

public class MyPropertyImplementation extends DefaultSinglePropertyImplementation<Concept, String> {

	public MyPropertyImplementation(ProxyMethodHandler<Concept> handler, ModelProperty<Concept> property)
			throws InvalidDataException {
		super(handler, property);
	}

    // Implements get: protocol
    // ('concatenate' semantics)
	@Override
	public void set(String aValue) throws ModelDefinitionException {
		String oldValue = get();
		if (get() != null) {
			super.set(aValue + get());
		}
		else {
			super.set(aValue);
		}
	}
}

public class MyListCardinalityPropertyImplementation<I, T> extends AbstractPropertyImplementation<I, List<T>>
		implements MultiplePropertyImplementation<I, T> {
	...
	// Implements get: protocol
	public List<T> get() {
		...
	}
	// Implements add: protocol
	public void addTo(T aValue) throws ModelDefinitionException {
		...
	}
	// Implements remove: protocol
	public void removeFrom(T aValue) {
		...
	}
}
\end{lstlisting}

Such mechanisms are really usefull to control and fine-tune code implementation. Suppose for example that you have a code used in monothreading context: all your multiple cardinality properties will rely for example on a \mytexttt{ArrayList} implementation. Instead of replacing all occurences of \mytexttt{ArrayList} by \mytexttt{Vector} (multi-thread safe version of Java \mytexttt{List}) in generated code, the developer has just one modication to do, in adequate \mytexttt{PropertyImplementation}.

\subsection{Multiple inheritance and traits programming}

\begin{lstlisting}[language=Java,basicstyle=\ttfamily\footnotesize]
@ModelEntity
public interface IntegerStorage extends AccessibleProxyObject {

     public static final String STORED_VALUE = "storedValue";

     @Getter(value = STORED_VALUE, defaultValue = "-1")
     public int getStoredValue();

     @Setter(value = STORED_VALUE)
     public void setStoredValue(int aValue);

     public void reset();

     @Implementation
     public abstract class IntegerStorageImpl implements IntegerStorage {
	@Override
	public void reset() {
	     setStoredValue(0);
	}

	@Override
	public void setStoredValue(int aValue) {
	     performSuperSetter(STORED_VALUE, aValue);
	     System.out.println("Sets stored value to be " + aValue);
	}
    }
}

@ModelEntity
public interface PlusProcessor extends IntegerStorage {

     public void processPlus(int value);

     @Implementation
     public abstract class PlusProcessorImpl implements PlusProcessor {
	@Override
	public void processPlus (int value) {
	     setStoredValue(getStoredValue()+value);
	     return getStoredValue();
	}
     }
}

@ModelEntity
public interface MinusProcessor extends IntegerStorage {

     public void processMinus(int value);

     @Implementation
     public abstract class MinusProcessorImpl implements MinusProcessor {
	@Override
	public void processMinus (int value) {
	     setStoredValue(getStoredValue()-value);
	     return getStoredValue();
	}
     }
}

@ModelEntity
public interface Calculator extends PlusProcessor, MinusProcessor {

}

\end{lstlisting}

\subsection{Embedding management and cloning support}

- embedding support
- cloning support (closure computation)
- clipboard operations

\subsection{Notification support}

- notification support
- isModified() support (embedding semantics)

\subsection{Persistance support}

- XML serialization/deserialization

\subsection{Graph computation features}

- support for computing equality
- diff/merge support, updateWith()
- visiting support



\subsection{Support for Contrat Programming}

- design by contract
- contract programming (JML)

% reference to jContractor

\subsection{Design patterns / Aspect programming / Code weaving}

- design patterns / aspect programming / code weaving








\section{PAMELA industrial use cases and experiments}
\label{sec:validation}

 The PAMELA framework has been successfully applied in a variety of complex programming and modeling scenarios and we continue to use it daily as part of our modeling toolbox. In the following, we describe three important uses cases in which PAMELA was a core component.

%\todo[inline]{General comment: I would change the title of the section to something like, Pamela Use cases, Pamela success stories or something of the like. The idea here is to show that pamela is a tool that is used and solve problems. In that sense, the fact that pamela powers Open flexo should be highlighted. In the same sense, the importance of the patterns and security should be reduced and limited to the fact that pamela allows it (the point being, to cite the workshop paper). I believe giving details of how patterns are implemented  here will distract the reader. technical details, if needed would be better placed in the implementation section.}

%https://pamela.openflexo.org
%https://github.com/openflexo-team/pamela

\subsection{Openflexo infrastructure}

Model Federation~\cite{Golra2016} is an approach that provides the means to
integrate multiple models conforming to different paradigms, and giving to each
stakeholder a specific view adapted to its needs. Model federation approach is
developed as a possible response to SIMF RFP (Semantic Information Modeling for
Federation)i~\cite{simf-rfp} by OMG (Object Modeling Group). This RFP (Request For Proposal)
requests submissions for a standard addressing, ”federation of information
across different representations, levels of abstraction, communities,
organizations, viewpoints, and authorities”. Thus model federation allows the
integration of heterogeneous models to develop new cross-concern
viewpoints/models or to synchronize the models used for designing a system. 

Openflexo\cite{OpenflexoWebSite} is a software infrastructure providing support
for model federation across multiple technological spaces. Conceptualization is
addressed trough the proposition of a language called FML (Flexo Modeling
Language), which is executable on the platform. Openflexo infrastructure
introduces connectors (also called Technology Adapters) to support various
technological spaces and paradigms.

This open source initiative is now mature at the infrastructure level, and many
projects and applications have been developed and powered using Openflexo
infrastructure. More than 15 technology adapters have been developed, with
various maturity stage regarding their industrialization (Microsoft Word, Excel
and PowerPoint, EMF, OWL, Diagramming, JDBC, XML, OSLC, etc...)

\sg{
The full Openflexo infrastructure is composed of about 50 components. In most components, PAMELA framework is extensively used. As an example \texttt{diana} component\footnote{\url{https://diana.openflexo.org}} (a component providing diagramming features) is composed of 998 classes. 159 of those classes (mostly the diagramming model) are defined as Pamela \texttt{ModelEntity}. 
}

The total base of code for Openflexo infrastructure represents around 900.000
lines of Java code. Regarding backend modeling, the PAMELA framework is
extensively used in Openflexo core as well as in most technology adapters, with
very few specific implementations for properties. The observer/observable
pattern is generally used for graphical user interfaces, which also rely on
PAMELA framework. 

\sg{
An interesting experimentation has been done in the context of Openflexo development process. When Pamela has been integrated to the code base, a big portion of former legacy code has gradually and iteratively been migrated to Pamela. Refactoring mainly consisted in removing code, and replacing method implementation by API method declaration. In some parts of core model implementation, code has been reduced by 80\% (in terms of lines of Java code), and many bugs disappeared, as they were caused by programming errors.
}

\sg{
Pamela implementation is now really stable and mature. According to Openflexo infrastructure developers, maintenance of code base (about 10 years of development) raises no Pamela-specific issue, while co-evolution of model and code is greatly improved compared to a "code generation"-based solution.
}

% Continuous integration ?

\subsection{Formose project}

The Formose ANR (french National Agency for Research) project
(ANR-14-CE28-0009)\cite{FormoseWebSite} aimed to design a formally-grounded,
model-based requirements engineering (RE) method for critical complex systems,
supported by an open-source environment. The main partners were: ClearSy, LACL,
Institut Mines-Telecom, OpenFlexo, and THALES. 

The main results of the project are a requirements modeling multi-views
language, its associated design process and the development of an opensource
platform called Formod\cite{FormodWebSite}, built using Openflexo
infrastructure and PAMELA framework. \todo{Say how and where PAMELA was used}
The requirements modeling language is based on KAOS~\cite{kaos} for goal
modeling and SysML for the structural part of a system. The associated domain
modeling language, used to describe system domain knowledge, extends the two
ontology languages OWL~\cite{owl} and PLIB~\cite{plib}. The graphical notations
are then translated into Event-B~\cite{eventb}, a formal specification method supported by verification tools. 

\todo{Remark that it is industry ready}

The Formose method and Formod tool have been evaluated on different case studies provided by the industrial partners of the project.



\subsection{SecurityPatterns experiment}

A significant experiment in PAMELA is the implementation of security patterns weaved on domain code \cite{silva20}.
In this context, the PAMELA framework is extended to include the notion of Pattern, i.e. a composition of multiple classes. Included to this experiment, the security pattern is specified by expected behavior defined and formalized by a pattern contract. This contract is defined by formal properties and the PAMELA framework ensures the property verification at runtime.

Related to the security pattern implementation, PAMELA enables the definition of additional security behavior to existing Java code.
Patterns are defined in PAMELA using three classes, each one representing a different conceptual level \texttt{PatternFactory}, \texttt{PatternDefinition}, \texttt{PatternInstance}.

To declare a Pattern on existing code, pattern elements such as Pattern Stakeholders and methods need to be annotated with provided security pattern-specific annotations. These annotations will be discovered at runtime by the \texttt{PatternFactory} and stored in \texttt{PatternDefinition} attributes.


Summarizing, implementing Patterns with PAMELA provides the ability to monitor the execution of the application code; the ability to offer extra structural and behavioral features, executed by the PAMELA interpreter; a representation of Patterns as stateful objects. Such objects can then evolve throughout runtime and compute assertions.





%\subsection{Performance issues}

% parler des problèmes de perfs

%\subsection{Syntaxic issues}

% Problème d'un truc trop verbeux > editeur graphique



\section{Related works}

\paragraph{Fame} 
\cite{kuhn2008fame}

\paragraph{Melanee}
\url{http://www.melanee.org/}
\cite{atkinson11}

\paragraph{Spoon}
\url{http://spoon.gforge.inria.fr}
\cite{pawlak2016}


\section{Conclusion}

The PAMELA framework promotes a modeling paradigm where models and code are jointly developed to provide a continuum between model and source code. The support is supplied by Java annotations and the PAMELA interpreter at runtime.

The different experiments provide efficient examples to argue the benefits of
the PAMELA framework but it is better to make your own experiences through {\url{https://pamela.openflexo.org/}.



%%
%% The next two lines define the bibliography style to be used, and
%% the bibliography file.
\bibliographystyle{ACM-Reference-Format}
\bibliography{biblio}

\end{document}
\endinput
%%
%% End of file `sample-sigchi.tex'.
