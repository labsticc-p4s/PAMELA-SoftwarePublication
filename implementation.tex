The PAMELA implementation is available online\footnote {\url{https://github.com/openflexo-team/pamela}}.

\subsection{Exposed API at design time}

The model-code integration we advocate requires facilities to encode metadata in the source code. This requires an annotation-enabled language. Such a language supports the attribute-oriented programming if its grammar allows adding custom declarative tags to annotate standard program elements. Java programming language is a good candidate, as it supports annotations.

The PAMELA API exposed to the developer mainly consists of : 1) a set of annotations; and 2) a set of unimplemented Java interfaces exposing required features.

The package \texttt{org.openflexo.pamela.annotations} package exposes the set of annotations which were presented in Subsection \ref{sub:DesignTime}.

The package \texttt{org.openflexo.pamela} contains the following feature-related Java interfaces:
\begin{itemize}
    \item \texttt{AccessibleProxyObject} is the interface that PAMELA objects should extend in order to benefit from base features such as generic default implementation, containment management, notification, object graph comparison and diff/merge, visiting patterns, etc.
    \item \texttt{CloneableProxyObject} exposes features related to cloning.
    \item \texttt{DeletableProxyObject} exposes features related to deletion management.
    \item \texttt{SpecifiableProxyObject} exposes dynamic assertion checking features in the context of JML (contract management) use. 
\end{itemize}

% Generic design patterns API, used in the context of aspect programming is exposed in \texttt{org.openflexo.pamela.patterns} package. A plug-in architecture allows to enrich model with some specific design pattern. Some basic design patterns are released with PAMELA 1.6.x in the context of security (Authenticator, Authorization, SingleAccessPoint, Owner).

\subsection{PAMELA interpreter}

The package \texttt{org.openflexo.pamela.factory} contains the PAMELA interpreter implementation. The core of the interpreter is implemented in the class \texttt{ProxyMethodHandler}.

From a technical point of view, the PAMELA implementation uses the \emph{javassist} reflection library (see \cite{shigueru2000}) which provides the \texttt{MethodHandler} mechanism, which is a way to override the Java dynamic binding. Invoking a method on an object which is part of a PAMELA model, causes the real implementation to be called when existing (more precisely dispatch code execution between all provided implementations), or the required interpretation according to the underlying model to be executed. This also provides an extension point allowing to instrument the code, which is used for other features such as undo/redo stack management, and assertion checking at runtime (support for Design by Contract, aka JML).

The PAMELA framework is a 100\% pure Java ($\geq$1.8), compilable by a classical Java compiler and executable in a classical Java Virtual Machine.

\subsection{PAMELA code base metrics}

The PAMELA implementation is modularized. 

\begin{itemize}
    \vspace{-0.2cm}\item Core implementation (\texttt{pamela-core}) provides all base features for PAMELA. It contains 20k lines of code involving 184 classes. This code is covered with unit tests (6k lines of code, 112 classes). Reached code coverage is about 66\%.
    \vspace{-0.2cm}\item \texttt{pamela-security-patterns} is an add-on library containing some security-pattern implementations.
    \vspace{-0.2cm}\item \texttt{pamela-perf-tests} contains benchmarking tools whose purpose is to quantify PAMELA performances. 
\end{itemize}

%\todo{Surely you can say more about its structure, size of components, ... other interesting and relevant observations on the codebase?}

\subsection{On performance issues}

Compared to a basic POJO (Plain Old Java Object) implementation, use of PAMELA involves a significant CPU and memory-footprint overhead, because of the partially-interpreted nature for some model features.

A workbench\footnote {\url{https://github.com/openflexo-team/pamela/tree/1.6.1/pamela-perf-tests}} has been developed to help measure that overhead. We use a base model composed of four entities defining 4 to 5 properties (single and multiple properties), and we instantiate that tree model (1010101 objects created and initialized to default value). From this simple model, two java implementations have been derived using code generation. The first implementation was made using fully implemented java classes (POJO), while the second one is fully interpreted (it uses the PAMELA framework and defines only interfaces and API methods). 

Unsurprisingly, we quantify a significant CPU overhead (from $\times$20). This is due to the fact that \texttt{MethodHandler} mechanism is fully interpreted and contains many hooks, and cannot be compared to a fully compiled and optimized Java dynamic binding. As well, memory footprint is increased from an 
$\times$5 factor.

These conclusions must be moderated by the fact that this overhead only applies
to a very small portion of the code (mostly property accessors methods). A "real-world" model implementation generally involves a bigger "business code" part, which is generally the first CPU-time consumer. Using PAMELA won’t increase CPU use on "business code" if this code is implemented with plain Java code (which is generally the case). As we will see in the next section, PAMELA has been successfully applied to many industrial projects, where no major performance issues were reported.

Moreover, comparison between the two implementations should be tempered by the fact that the PAMELA implementation offers many functional features which are not present in the base implementation (such as undo/redo, clipboard management, runtime monitoring and weaving, etc.).

% Rajouter un paragraphe sur le fait que PAMELA est une implementation du MOP (meta object protocol) ?

% Is it the good place for following links ?
%https://pamela.openflexo.org
%https://github.com/openflexo-team/pamela
