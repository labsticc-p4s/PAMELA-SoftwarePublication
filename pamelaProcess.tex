We advocate for a strong coupling between model and sources code, to give architects and developers a way to both interact during the whole development cycle. PAMELA is an annotation-based Java modeling framework providing a smooth integration between model and code, without code generation nor externalized model serialization. Instead of generating the code, the API (mostly Java interfaces with abstract method declarations) is locally executed (interpreted). The idea is to avoid separation between modeling and code to facilitate consistency management and to avoid round-trip issues. Figure \ref{fig:PamelaVision} summarizes the PAMELA architecture and is more detailed thereafter.  

\begin{figure}
    \centering
    \includegraphics[width=1.0 \columnwidth]{PamelaVisionV2.pdf}
    \caption{PAMELA approach for modeling}
    \label{fig:PamelaVision}
\end{figure}

%\subsection{PAMELA Modeling/Development Process}

Coupling model and code into the same artifact opens new ways of programming. The classical (metadata enabled) programming process relies on \emph{programmers} that produce code reusing pre-existing modeling concepts. These concepts are implemented by \emph{modelers} that provide the right annotations the programmers use. This is, for instance, the process followed by Jakarta EE (JEE) developers reusing JEE specific annotations. 
% Joel : une ref pour Jakarta ?
The evolution rhythm between models and code is low. This programming way is still possible with PAMELA, but we allow the ability to reach a high evolution rhythm when the programmer also becomes the modeler. In fact, when the programmer identifies a pattern, an abstraction, a generalization, s/he can use PAMELA to develop and capitalize on this abstraction by extending PAMELA's metamodel. 
% Joel : le MM de Pamela n'est pas étendu en tant que tel (=M3) c'est le niveau M2 qui étendu ??? mais la bibliothèque de metamodèles est étendue ??? donc juste mettre PAMELA's metamodels. ????
% Je pense qu'ici il faudrait plutot parler du modèle. Le métamodèle PAMELA est celui représenté figure 3. Ou alors parler du metametamodele PAMELA

The developed metamodels are implemented by annotations that rely on Java/JVM entities and mechanisms. They include consistency checking, which constrain their use and help the programmer to avoid inconsistencies or errors. We have first experimented their use with setters/getters to define Plain Old Java Objects (POJO), with traits to implement multiple inheritance or with roles and rules to set security rules on classes.

Our experience shows that introducing and reusing new concepts (1) reduce the size of the code (2) reduce the risk of errors and (3) improve the code structure. The cycle of development between the model and the code can then be drastically reduced, leading to what we call \emph{continuous modeling}.

The code size is reduced because abstractions factorize recognized concepts so that the code using such concepts is replaced by the use of the abstraction at the right place. This also reduces the risk of errors since the code is now managed by the PAMELA framework with all the required checks. Finally, the code structure is improved since it matches the way the programmer conceptualizes (models) her/his code.

Here are various conceivable scenarios for PAMELA use:
\begin{itemize}
    \vspace{-0.2cm}\item \textbf{Programming use} : programmers reuse existing annotations and write model and code at the same time (the modelers and programmers share the same artifacts : the Java code). This scenario includes the case where the model (made by the modelers) is pre-existing.
    \vspace{-0.2cm}\item \textbf{Reengineering} : programmers start from an existing code base (legacy) and refactor it while replacing this code by abstract method declarations in Java interface, reducing risks of errors.
    \vspace{-0.2cm}\item \textbf{Aspect-oriented programming} : programmers may use or redefine "patterns" (e.g., Security Patterns \cite{silva20}) which offers code weaving at runtime, and runtime monitoring.
    \vspace{-0.2cm}\item \textbf{Advanced programmers use} : programmers may extend PAMELA with their own annotations, implementations or patterns.
\end{itemize}

In this presentation, we focus on the programming use of PAMELA, when programmers reuse existing annotations. The full power of PAMELA arises when programmers become modelers defining their own abstractions/annotations.

