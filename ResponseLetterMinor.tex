% Title:    A LaTeX Template For Responses To a Referees' Reports
% Author:   Petr Zemek <s3rvac@gmail.com>
% Homepage: https://blog.petrzemek.net/2016/07/17/latex-template-for-responses-to-referees-reports/
% License:  CC BY 4.0 (https://creativecommons.org/licenses/by/4.0/)
\documentclass[10pt]{article}

% Allow Unicode input (alternatively, you can use XeLaTeX or LuaLaTeX)
\usepackage[utf8]{inputenc}
\usepackage{hyperref}
\usepackage{xcolor}
\usepackage{todonotes}

\usepackage{microtype,xparse,tcolorbox}
\newenvironment{reviewer-comment }{}{}
\tcbuselibrary{skins}
\tcolorboxenvironment{reviewer-comment }{empty,
  left = 1em, top = 1ex, bottom = 1ex,
  borderline west = {2pt} {0pt} {black!20},
}
\ExplSyntaxOn
\NewDocumentEnvironment {response} { +m O{black!20} } {
  \IfValueT {#1} {
    \begin{reviewer-comment~ }
      \setlength\parindent{2em}
      \noindent
      \ttfamily #1
    \end{reviewer-comment~ }
  }
  \par\noindent\ignorespaces\color{blue}
} { \bigskip\par }

\NewDocumentCommand \Reviewer { m } {
  \section*{Comments~by~Reviewer~#1}
}

\NewDocumentCommand \ToolComments { m } {
  \section*{Tool~Comments~by~Reviewer~#1}
}

\ExplSyntaxOff
\AtBeginDocument{\maketitle\thispagestyle{empty}\noindent}

% You can get probably get rid of these definitions:
\newcommand\meta[1]{$\langle\hbox{#1}\rangle$}
\newcommand\PaperTitle[1]{``\textit{#1}''}

\title{Response letter to Manuscript SCICO-D-20-00176R1 entitled "PAMELA: an Annotation-Based Java Modeling Framework"}
%\author{Author1 \and Author2 \and Author3}
\date{\today}

\begin{document}

Dear Editor,

\bigskip
Thanks for your message of Mars 15th, 2021, in which you suggested preparing a minor revision of our manuscript  \textbf{PAMELA: an Annotation-Based Java Modeling Framework}. 

\bigskip
We are grateful to the reviewers for their interest, effort, and suggestions. We detail in the following pages how we have addressed the reviewers' concerns;  We have organized the responses to the comments on a per-reviewer basis, focusing, as suggested, on: 

\begin{itemize}
\item Fixing typos, language errors, and other minor issues.
\item Refactoring Section 2.
\item Provide further explanations with respect to the performance of the approach.
\end{itemize}

Moreover, the paper has been thoroughly revised to fix minor issues and to give the additional clarifications required by the reviewers. Please let us know if you need any further information or clarification.

\bigskip
Sincerely yours,

\bigskip
Sylvain, Guillaume, Caine, Joel, Jean-Christophe, Salvador, Fabien and Antoine

\pagebreak


\Reviewer{\#1}
\begin{response}{There remain or are introduced however, a substantial number of typos, language errors, and other minor issues that need correcting:} 

All highlighted typos, languages errors and minor issues have been checked and fixed.

%\begin{itemize}

%\item "to /avoid" on p3 (DONE)
%\item "detailed thereafter" $->$ "detailed hereafter" (It's Correct : As adverbs the difference between hereafter and thereafter is that hereafter is in time to come; in some future time or state while thereafter is after that, from then on.)
%\item "increasing PAMELA metamodel" $->$ "by \_extending\_ PAMELA\_'s\_ metamodel" (DONE)
%\item "with setter/getter": make both plural (DONE)
%\item POJO used here and elsewhere yet the acronym is introduced much later (DONE)
%\item p4 l11-12: reduce etc. should be reduce\_s\_ etc. (Keep as it is)
%\item odd spacing before : consistently throughout bullet list + text  (Keep as it is: trust Latex)
%\item "in the section 3.2" $->$ "in Section 3.2" (style: named references deserve a capital and no article before) (DONE)
%\item p5 and many times: "PAMELA metamodel", "PAMELA model", "PAMELA framework" need "the" before them. (Done ; with Pamela $->$ PAMELA)
%\item "defines API": insert "an" (DONE)
%\item "in Java interface": insert "a" (DONE)
%\item "returning desired value", "taking value" etc.: insert "the" (DONE)
%\item "May also declares"... (DONE)
%\item p6 l46: @Embedded/"as embedded" meaning?
%\item p7 "of launched Java application" missing "the" (DONE)
%\item "is/are" $->$ "is (are)", in line with parentheses earlier in sentence (DONE)
%\item "Experimented programmers": I doubt it, unless you got major ethical clearance (Replaced by "Experienced PAMELA programmers")
%\item "Listing 2: model" $->$ "~: Model" consistent with Listing 1 (DONE)
%\item the book title is "Lord of the ring\_s\_" with s (DONE)
%\item "using classical Java ... scheme": missing "the" (DONE)
%\item "Java programming language": insert "The" (DONE)
%\item l43 inconsistent spacing (DONE)
%\item l46 package ... package (DONE)
%\item p11 l36 sticking into margin (Final check to do)
%\item S.4.3 please add size in LoC and classes (?)
%\item java $->$ Java (DONE)
%\item p13 "diana component": insert "the" (DONE)
%\item "experimentation" $->$ "experiment" (DONE)
%\item "When Pamela has been" $->$ "After Pamela was" (DONE : if we consider it's over)
%\item l36 the quotation mark before code generation is the wrong one 'changed to ' and ' ?)
%\item p 14 "appertaining" $->$ "pertaining", "relating" (Kept as is. As verbs the difference between pertain and appertain is that pertain is to belong while appertain is to belong to or be a part of, whether by right, nature, appointment, or custom; to relate to.)
%\end{itemize}

\end{response}

%\pagebreak

\Reviewer{\#2}
\begin{response}{Would it be possible to structure it more, because currently it is several paragraphs the one bellow the others.}

Section 2 has been decomposed into two Subsections: Subsection 2.1.  Architecture overview
and Subsection 2.2.  Usage of PAMELA.

\end{response}


\begin{response}{figure 2 is not enough described in section 2 and even in the paper.}

The end of Subsection 2.1 provides now a full explanation of figure 2.

\end{response}

\begin{response}{More over, concerning section 4.4, I appreciate this new section. However, while reading, it is not a good advertisement for Pamela. Overhead (x20) is huge. Perhaps it is not the right benchmark if it provides bad results that are not representing the right usage.}

We agree with the reviewer that the overhead may seem huge. Note, however, that these measures correspond to the very worst-case scenario: an application with absolutely no business logic. Although such applications do not correspond to the normal usage, we believe showing this measurement is important to fully characterize the performance of the approach. E.g., PAMELA should not be used when the application is a mere data structure.

For normal applications, the time passed on attribute accessors is negligible compared to the time spent on business logic. We have verified this by using a profiling tool on PAMELA based applications (we have added this information to the paper). Moreover, as we show in Section 5 PAMELA has been successfully applied to many industrial projects, with no performance issues reported. 

\end{response}

\begin{response}{Minors}
We have fixed all the highlighted minor issues.
\end{response}

\end{document}
