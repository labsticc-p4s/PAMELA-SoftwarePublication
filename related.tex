Very few approaches build interfaces between MDE artifacts and source code at runtime. FAME \cite{kuhn2008fame} is  a library that keeps metamodels accessible and adaptable at runtime. Basically, FAME attaches meta-information to the objects of a running application permitting an uniform manipulation of both models and system objects at runtime. Similarly, in \cite{song2010applying} the authors use runtime models to maintain a causal connection between a model and a running systems for monitoring and control purposes. More recently, in \cite{boronat2019} the authors construct and maintain at runtime model-based views on the data manipulated in object-oriented code. Different to the aforementioned approaches, PAMELA does not focus in reflecting and manipulating running systems but in: 1) extending their behaviors; and 2) maintaining a continuous cycle of metamodeling so that models and corresponding code remain synchronized without the need for code generation processes. In a different approach, UMPLE \cite{lethbridge2016merging} mixes programming and modeling by integrating UML constructs into languages such as Java. However, they use code generation for the runtime part of the system.